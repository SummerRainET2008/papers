\begin{abstract}
 Task-specific chatbot systems have gained many important applications, such as smart speaker, customer service system. 
 One fundamental module behind them is detecting intent of a user's input, and can be modeled as a short text classification problem. 
 However, in the early stage of building a chatbot, collecting enough labeled data for hundreds of thousands of user intents is expensive. 
 Popular classification models, direct mapping a query to an intent, have a high precision, while depending on enough task-specific labels information. 
 In comparison, similarity models, modeling similarity of two queries instead, can utilize additional out-of-domain data, while having a relatively lower precision, due to the discrepancy of similarity loss and real classification loss. 
 In this work, we propose a novel model, called similarity model fused with classification model (SFC), to combine the merits of the two kinds of models in the framework of multi-task training. 
 Our extensive experiments on 6 public and 1 private datasets demonstrate that our systems outperform very strong baselines (i.e., RoBERTa based pretrained model, joint model with NER), especially with insufficient data.

\end{abstract}

