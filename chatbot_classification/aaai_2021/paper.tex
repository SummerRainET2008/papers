%% LyX 2.3.1-1 created this file.  For more info, see http://www.lyx.org/.
%% Do not edit unless you really know what you are doing.
\documentclass{article}
\usepackage[latin9]{inputenc}
\usepackage{array}
\usepackage{multirow}
\usepackage{amsmath}
\usepackage{graphicx}

\makeatletter

%%%%%%%%%%%%%%%%%%%%%%%%%%%%%% LyX specific LaTeX commands.
%% Because html converters don't know tabularnewline
\providecommand{\tabularnewline}{\\}
%% A simple dot to overcome graphicx limitations
\newcommand{\lyxdot}{.}


%%%%%%%%%%%%%%%%%%%%%%%%%%%%%% User specified LaTeX commands.



%%%%%%%%%%%%%%%%%%%%%%%%%%%%%% User specified LaTeX commands.
% Template for GlobalSIP-2018 paper; to be used with:
%          spconf.sty  - ICASSP/ICIP LaTeX style file, and
%          IEEEbib.bst - IEEE bibliography style file.
% --------------------------------------------------------------------------
\usepackage{spconf}


\def\x{{\mathbf x}}
\def\L{{\cal L}}

\makeatother

\begin{document}
\title{     
  SFC: Few-Shot Text Classification via Similarity Fused with Classification System
}
\name{Jingwen Huang}   
\address{     
  \{hanmei613\}@gmail.com   
}
\maketitle 

\begin{abstract}
  Building  conversational chatbot system has become a popular solution to sharing
  the  work  of customer service under various business scenarios. One of the most
  important  applications is the single-turn conversational chatbot aiming at user
  intent  detection, which essentially equals to short-text classification problem
  in  the  field  of  Natural  Language  Processing.  However, each time for a new
  service,  the  chatbot system often needs to perform well in few-shot setups due
  to  lack  of  domain-specific  data,  which  is  still quite hard even if we use
  powerful  pretrained  model  like  Roberta. Therefore, in this paper, we propose
  SFC,  a  system  fusion  of  both  similarity  model and classification model to
  overcome  this  challenge.  Our main contributions are: 1) transfer learning and
  negative  sampling  based  on  sentence-pair  are utilized as remedy for lack of
  data;  2)  multi-task  learning is involved to achieve faster training speed and
  better  performance;  3) model ensembling of classification and similarity model
  guarantees  inference  speed  while keeping high accuracy. Additionally, we also
  conduct  extensive  experiments on four public datasets in few-shot setup (i.e.,
  with  only  5 to 20 training data per class). The experimental results show that
  our system can steadily outperform several competitive baselines by 2 percent in
  average accuracy.
\end{abstract}
\keywords{stress, detection, attention, neural network}

\section{Introduction}
\label{sec:intro}

Single-turn  conversational  chatbots  are  designed to transform existing tasks
that  rely  on human agents, such as classifying customers' questions or queries
to  find  the  corresponding  answer,  into an automatic process based on intent
classification   model.  Since  user  queries  are  usually  much  shorter  than
paragraphs  or  documents,  the chatbot can actually be turned into a short-text
Classification  task  in  Natural  Language Processing. Moreover, at the initial
stage  of  building  chatbots, it's usually extremely hard to collect sufficient
data  for each class. In this way, building a single-turn conversational chatbot
become  a  short-text  classification  problem under few-shot setting(Yu et al.,
2018).

Short  texts are usually more ambiguous in comparison with long texts since they
don't contain enough contextual information, which poses a great challenge for
classification(Chen et al., 2019). In addition, the few-shot scenario(Yu et al.,
2018)  adds  even more difficulties to the classification task since there is no
enough  information  for  the  model  to  learn  for  each  class.  Comparing to
non-pre-trained neural network such as convolutional neural networks (CNNs)(Kim,
2014) and long short term memory networks (LSTMs)(Mousa and Schuller, 2017), the
recently introduced pre-trained language models on large corpus like BERT(Devlin
et  al.,  2019)  and  RoBERTa(Liu et al., 2019) has been proven more powerful in
solving  many  NLP  tasks  including  short-text  classification  for  deficient
data(Madabushi,  H.  T.,  2020).  Especially  for  few-shot  scenarios, transfer
learning  based  on  pre-trained  model  tends  to  do more help to the negative
effects brought by scarcity of training data. Due to the fact that RoBERTa is an
improved  version  of  BERT,  we  build  our SFC chatbot system using RoBERTa as
pre-trained  context-dependent  embeddings.  To  our  knowledge, the most common
approach  is  adding  a  softmax  classifier to the top of RoBERTa(i.e., RoBERTa
classifier),  which  turns  the  problem into a simple text classification task.
However,  despite  the  fact  that  RoBERTa has achieved amazing results in many
Natural  Language  Processing tasks, we still believe it has much more potential
under few-shot setting.

Therefore,  in  this  paper  we  propose  to  involve  similarity  model  (i.e.,
sentence-pair  classifier)  based  on  RoBERTa,  which  can  score  the semantic
similarity between two sentences by multiple cross-attention mechanism(Barkan et
al.,  2019).  A natural idea to enhance model performance in few-shot setting is
to  further  pre-train RoBERTa with target domain data for transfer learning(Sun
et  al.,  2019).  However, it's always not easy to find domain-specific data for
certain  service/product  if  we  try  to further pre-train a RoBERTa classifier
directly.  In  comparison,  it's  relatively  feasible  to  obtain  dataset  for
semantically duplicate sentence pair identification task. That is to say, we can
obtain a further pre-trained similarity model to identify the class with highest
semantic  similarity  level to each user query. Afterwards, we can fine-tune the
similarity  model using sentence pairs sampled from target task dataset based on
negative  sampling  strategy(Bamler  and Mandt, 2020), which can provide us with
more features to learn from limited amount of data.

We  also apply multi-task learning(Caruana, 1993; Collobert and Weston, 2008) in
training  process  to  enhance the training speed and performance. We set up two
different  objectives  for  training. The first one is the regular sentence-pair
similarity  score  which  help the model learn the semantic similarity between a
query  to each existing data point. The second target is to learn the similarity
score  between  a  query  and all the data points of a certain class as a whole.
Specific  knowledge contained in these two tasks can be fully explored to obtain
a faster training speed and higher accuracy.

Another  contribution of this work is that we ensemble the similarity model with
classification  model  at  the  inference  step.  In  contrast with common model
ensembling  methods,  the classification model are used as an auxiliary model in
our  system  to  select promising sentence-pair candidates for similarity model.
The  experiment  results  on  4  short-text  classification  datasets in various
few-shot  settings  show that our system can outperform single model baseline by
at  least  2 percent on average. Besides, we can also control the inference time
for one single query to be within 0.5 seconds, which makes the system applicable
in real-life single-turn chatbot scenario.

\section{Attention based neural network}

\section{Conclusion}

In this work, we successfully applied two widely used attention techniques
into lexical stress detection problem. Our proposed model directly
takes time series features as input that it could fully explore input
information; and the network structure instrinstically support words
with different number of syllables, without expanding short words,
to reduce input approximation. Experiments also demonstrate the improvements
over baselines systems.

\bibliographystyle{IEEEbib}
\bibliography{refs}

\end{document}
