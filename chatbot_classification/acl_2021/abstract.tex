\begin{abstract}
  Task-specific chatbot systems have gained many important applications, such as
  smart  speaker,  customer  service systems. One essential technique, detecting
  intention behind a user's input, can be modeled as a short text classification
  problem.  In  the early stage of buidling a chatbot, collecting enough labeled
  data  for  hundreds  of  thousands  of  user  intention  is expensive. Popular
  classfication models, direct mapping query to intention, have a high precsion,
  while depending on enough task-specific labels information. Similarity models,
  modeling  similarity  of two queries instead, can utilize out-of-domain data,
  while  having  a  relatively  lower  precision,  due  to  the  discrepency  of
  similairty loss and real classification loss. In this work, we propose a novel
  model,  called  similarity  model  fused  with  classification model (SFC), to
  combine  the  merits of the two kinds of models in the framework of multi-task
  training.  Our  extensive  experiments  on  6  public  and  1 private datasets
  demostrate  that  our  systems outperform very strong baselines (i.e., RoBERTa
  based  pretrained  model,  joint model with NER), espeically with insufficient
  data.

\end{abstract}

